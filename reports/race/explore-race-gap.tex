\documentclass[12pt,]{article}
\usepackage{lmodern}
\usepackage{amssymb,amsmath}
\usepackage{ifxetex,ifluatex}
\usepackage{fixltx2e} % provides \textsubscript
\ifnum 0\ifxetex 1\fi\ifluatex 1\fi=0 % if pdftex
  \usepackage[T1]{fontenc}
  \usepackage[utf8]{inputenc}
\else % if luatex or xelatex
  \ifxetex
    \usepackage{mathspec}
  \else
    \usepackage{fontspec}
  \fi
  \defaultfontfeatures{Ligatures=TeX,Scale=MatchLowercase}
\fi
% use upquote if available, for straight quotes in verbatim environments
\IfFileExists{upquote.sty}{\usepackage{upquote}}{}
% use microtype if available
\IfFileExists{microtype.sty}{%
\usepackage{microtype}
\UseMicrotypeSet[protrusion]{basicmath} % disable protrusion for tt fonts
}{}
\usepackage[margin = 1.25in]{geometry}
\usepackage{hyperref}
\PassOptionsToPackage{usenames,dvipsnames}{color} % color is loaded by hyperref
\hypersetup{unicode=true,
            pdftitle={Memo: The Racial Voting Gap and the Democratic Vote},
            pdfauthor={Michael G. DeCrescenzo},
            colorlinks=true,
            linkcolor=red,
            citecolor=black,
            urlcolor=blue,
            breaklinks=true}
\urlstyle{same}  % don't use monospace font for urls
\usepackage[style=chicago-authordate]{biblatex}

\addbibresource{/Users/michaeldecrescenzo/Dropbox/bib.bib}
\usepackage{longtable,booktabs}
\usepackage{graphicx,grffile}
\makeatletter
\def\maxwidth{\ifdim\Gin@nat@width>\linewidth\linewidth\else\Gin@nat@width\fi}
\def\maxheight{\ifdim\Gin@nat@height>\textheight\textheight\else\Gin@nat@height\fi}
\makeatother
% Scale images if necessary, so that they will not overflow the page
% margins by default, and it is still possible to overwrite the defaults
% using explicit options in \includegraphics[width, height, ...]{}
\setkeys{Gin}{width=\maxwidth,height=\maxheight,keepaspectratio}
\setlength{\emergencystretch}{3em}  % prevent overfull lines
\providecommand{\tightlist}{%
  \setlength{\itemsep}{0pt}\setlength{\parskip}{0pt}}
\setcounter{secnumdepth}{5}

%%% Use protect on footnotes to avoid problems with footnotes in titles
\let\rmarkdownfootnote\footnote%
\def\footnote{\protect\rmarkdownfootnote}

%%% Change title format to be more compact
\usepackage{titling}

% Create subtitle command for use in maketitle
\newcommand{\subtitle}[1]{
  \posttitle{
    \begin{center}\large#1\end{center}
    }
}

\setlength{\droptitle}{-2em}

  \title{Memo: The Racial Voting Gap and the Democratic Vote}
    \pretitle{\vspace{\droptitle}\centering\huge}
  \posttitle{\par}
    \author{Michael G. DeCrescenzo\footnote{Ph.D.~Candidate, Political Science,
  University of Wisconsin--Madison}}
    \preauthor{\centering\large\emph}
  \postauthor{\par}
      \predate{\centering\large\emph}
  \postdate{\par}
    \date{Updated December 13, 2018}

% ----------------------------------------------------
%   a preamble file for use with R Markdown
% ----------------------------------------------------

% --- text and typography -----------------------

% not actually sure that microtype works!
% \usepackage{microtype}   % microtypography

\usepackage{mathptmx} % serif = times (with math)
\usepackage{helvet} % sens-serif = helvetica clone
\usepackage[varqu, scaled = 0.95]{zi4}             % mono w/ straight quotes

% are these needed??
% \usepackage[utf8]{inputenc} % better interpretation of input characters
% \usepackage[T1]{fontenc}    % better output glyphs/behaviors
% I'm not sure how well these two packages work either!


\usepackage{bm}

% --- floats -----------------------
% is this needed??
% \usepackage{booktabs}
% \usepackage{longtable} % for selective float-prevention
\usepackage{dcolumn} % decimal-aligned columns
\usepackage{rotating}
\usepackage{placeins}


% --- Styles -----------------------

% TITLE
\usepackage{titling}

  % title field
  \pretitle{\begin{center} \LARGE}
  \posttitle{\par\end{center}\vskip 12pt}

  % author field
  \preauthor{\begin{center}\large}
  \postauthor{% \\ \normalsize 
              % \emph{University of Wisconsin--Madison}
              \par\end{center}}

  % date field
  \predate{\begin{center}}
  \postdate{\par\end{center}}

% ABSTRACT
\usepackage{abstract}
  \renewcommand{\abstractname}{}    % clear the title
  \renewcommand{\absnamepos}{empty} % originally center

% % SECTIONS
\usepackage[small, bf]{titlesec}
  % \titleformat*{\subsection}{\itshape}
  \titleformat*{\subsubsection}{\itshape} 
  \titleformat*{\paragraph}{\itshape} 



% --- Other document logic -----------------------

% to-do notes
\usepackage[colorinlistoftodos, 
            prependcaption, 
            obeyFinal,
            textsize = footnotesize]{todonotes}
  \presetkeys{todonotes}{fancyline, color = violet!30}{}


% --- bib -----------------------

% natbib declared in RMD
% \usepackage{natbib}
 % \bibpunct[: ]{(}{)}{;}{a}{}{,}
\usepackage[lf, mathtabular, minionint]{MinionPro} % serif
\usepackage{MyriadPro}                             % sans family
\usepackage[varqu, scaled = 0.95]{zi4}             % mono w/ straight quotes

\usepackage{amsthm}
\newtheorem{theorem}{Theorem}[section]
\newtheorem{lemma}{Lemma}[section]
\theoremstyle{definition}
\newtheorem{definition}{Definition}[section]
\newtheorem{corollary}{Corollary}[section]
\newtheorem{proposition}{Proposition}[section]
\theoremstyle{definition}
\newtheorem{example}{Example}[section]
\theoremstyle{definition}
\newtheorem{exercise}{Exercise}[section]
\theoremstyle{remark}
\newtheorem*{remark}{Remark}
\newtheorem*{solution}{Solution}
\begin{document}
\maketitle
\begin{abstract}
This document applies the method laid out in
\textcite{burden-decrescenzo-gap} to the political cleavage of race
(White and Nonwhite). This case is of interest because the groups are of
vastly different sizes, so the benefit of targeting one group over the
other is obscured.
\end{abstract}

\hypertarget{exposition}{%
\section{Exposition}\label{exposition}}

We begin with data that codes each voter along race (white or nonwhite),
party affiliation (Democrat, Republican, or unaffiliated) and vote
choice (Democrat, Republican, or other). This results in \(18\) possible
outcome categories in each election, each with an associated probability
of incidence in the electorate.

As before, we estimate these probabilities with uncertainty using Markov
chain Monte Carlo. MCMC generates a distribution of parameter estimates
that are compatible with the model and the data. The ``posterior''
distribution of samples reflects a flat Dirichlet prior and thus should
be proportional to the likelihood of the data.

We run the sampler on four chains with 2,000 iterations per chain,
setting aside the first 1,000 that are used for adaptive warmup.
Following \textcite{Link2011}, we elect not to thin the parameter
chains.

\hypertarget{model-estimates}{%
\section{Model Estimates}\label{model-estimates}}

This exercise would benefit from one figure that we don't include: a
comparison of the size of the White and Nonwhite groups in the total
electorate. We would do this by summing all of the ``White'' and
``Nonwhite'' probabilities, respectively, to show the fraction of the
electorate in each group.

\begin{figure}

{\centering \includegraphics[width=1\linewidth]{figs/plot-rhs-1} 

}

\caption{Right-hand side terms}\label{fig:plot-rhs}
\end{figure}

Showing group totals would help explain the results in
Figure~\ref{fig:plot-rhs}, where the share of white Democrats is falling
precipitously, but white Republicans isn't gaining nearly as fast. This
is because the nonwhite share of the electorate is growing while white
voters are converting from Democrats to Republicans. The racial
composition of Democratic partisans changes dramatically, but the
distribution of partisanship in the electorate is barely affected at the
high level.

\FloatBarrier

\hypertarget{creating-the-race-gap}{%
\section{Creating the Race Gap}\label{creating-the-race-gap}}

\begin{sidewaysfigure}

{\centering \includegraphics[width=1\linewidth]{figs/plot-gap-1} 

}

\caption{Effects on Gap}\label{fig:plot-gap}
\end{sidewaysfigure}

Figure~\ref{fig:plot-gap} shows how Democratic advantages in each
mechanism affect the racial voting gap. Democratic advantage is
positively related to the gap for Nonwhites, and negatively related for
Whites.

Nonwhites' contribution to the gap is essentially partisanship (mostly
population growth?). Whites contribute the gap primarily by a higher
Republican mobilization and persuasion advantages in the early years (a
positive effect on the gap that diminishes over time), and a strong
partisan drift across the entire series (a negative effect that becomes
a positive effect). The final panel shows that a race gap has existed
throughout the entire series, but the underlying mechanisms for it have
changed dramatically over time.

\FloatBarrier

\hypertarget{creating-the-democratic-vote}{%
\section{Creating the Democratic
Vote}\label{creating-the-democratic-vote}}

\begin{sidewaysfigure}

{\centering \includegraphics[width=1\linewidth]{figs/plot-vote-1} 

}

\caption{Effects on Vote}\label{fig:plot-vote}
\end{sidewaysfigure}

Figure~\ref{fig:plot-vote} shows how the Democratic advantage in each
mechanism affects the vote. The race gap is positive when Democrats net
more votes from Nonwhites than they do from Whites.

In partisanship, we can see the Democratic advantage among Nonwhites
slowly growing, while Democratic support among whites shifts and turns a
``negative'' race gap into a positive one.

Mobilization shows the advantage among non-mobilized votes. We find that
the numerical impact of Nonwhite mobilization is mostly flat. This means
that Democrats lose more nonwhite voters than Republicans in an an
absolute number, but there are more Democratic voters to start with. The
stability in the trend is striking; if we assume a fixed
non-mobilization rate among Nonwhite Democrats, then the number of lost
votes should increase as the number of Nonwhite Democrats. Instead, we
find that non-mobilization stays constant even as the number of
partisans increases. This could indicate that there are two forces in
equilibrium: an increasing number of Nonwhite Democrats and increasingly
turnout among Nonwhites at the same time. We would want to divide the
proportion of Nonwhite Democratic voters by the proportion of Nonwhite
Democratic identifiers to get the turnout rate among Nonwhite Democrats.

The final panel clearly shows the trade-off between White and Non-white
votes. Democrats have nearly always netted more votes from Nonwhite
voters than from White voters, but since 2008 has Democratic performance
among Nonwhites \emph{made up} for their losses among Whites. The racial
voting gap has grown gradually from the growth of Nonwhite Democratic
partisans, but acute change in the gap is driven largely by short-term
movement among Whites.

\printbibliography


\end{document}
